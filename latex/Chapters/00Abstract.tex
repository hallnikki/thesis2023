\noindent The challenges faced by the urban environment during the recent pandemic must be studied in order to be prepared for the next global health crisis. During the first weeks of the COVID-19 pandemic, cities around the world instituted stay-at-home orders and lockdowns of varying degrees, forcing their residents to adapt by connecting online to fulfill social needs. However, data collected during this period makes evident that these digital stop-gaps are no substitute for face-to-face social interactions. As the pandemic’s duration continued to exceed all expectations, public health officials could have encouraged the use of outdoor spaces with appropriate social distance. Socially-distanced outdoor gatherings give people the means to connect in-person while maintaining a low infection risk. But how many people live near an outdoor public space that would allow for these social gatherings?

\noindent This research uses aerial imagery, government census statistics, and open source geospatial data with geographic information system software and network analysis methods to evaluate whether large city parks in Tokyo, London and New York can support social gatherings in a pandemic era. This thesis compares and analyzes Tokyo’s Yoyogi Park, London’s Hyde Park and New York’s Prospect Park and their respective surrounding neighborhoods, park entry points, and road networks to determine the extent to which social gatherings are currently accessible. Moreover, the research aims to discover a metric to work towards when making changes to existing parks in preparation for the next pandemic. In short, the data shows that even these internationally-known city parks have difficulty providing a sufficient number of gathering spaces for the residents who live within a one-kilometer radius, raising questions about how parks could change before the next public health crisis begins.








