\noindent My fascination with the urban fabric of Tokyo began almost ten years ago when I took my first trip to Japan as an undergraduate architecture student. Now, in 2023, I am fortunate enough to spend my days researching and writing about this city while reflecting on my experiences in other cities, in my home country (the United States) and beyond.

\noindent When I started this master's program at the University of Tokyo in fall 2021, Japan's borders were closed to international students due to the COVID-19 pandemic. Because I spent the first eight months of school in San Francisco but longing to live in Tokyo, I could not help but begin my research from a place of comparison. San Francisco is a dense, walkable city, but in the first few months of the pandemic when public transportation was suddenly limited and discouraged, the fact that the nearest supermarket was over a mile away from home became a major challenge. While friends and family in other areas of the US would drive their car to the store, load up on groceries for two to three weeks and then retreat to their isolated homes, we had no choice but to make frequent trips to the nearest store within walking range and then carry as much as we could back up the steep hill to our apartment. As the initial fear and wave of COVID-19 cases eased, we began meeting friends in Golden Gate Park (closer in distance to our residence than the supermarket) where we would spend hours sitting 6 feet (2 meters) apart with food we had picked up (almost) along the way. Our apartment had no outdoor space, so this was the only way to engage with friends in a way that felt safe during the first six months of the pandemic. I often wondered if people living in Tokyo would have had the opposite experience, with better access to groceries and less access to parks.

\noindent Reflecting on all we learned and lived through during the pandemic, many of the resulting changes  continue to be felt. In Japan, masks are still commonplace, disinfecting alcohol spray sits at the entrance to every store, and many public places post signs recommending social distancing, which is extremely difficult to do in a densely populated city. In the US, the pandemic-inspired migration from cities to less dense areas of the country is reminiscent of the suburban exodus in the mid-twentieth century, a trend which leaves cities with complex issues in terms of infrastructure as well as socially and politically. 

\noindent Using parks for social interaction during a pandemic is the starting point from which I will explore the relationship between walkability, neighborhood amenities, and the parks where social interactions can occur during a pandemic. My research aims to address how cities  might prepare for future pandemics, but as I write this thesis I have just experienced my first cherry blossom season in Japan. Walking through the parks where thousands of people had set up blankets and brought food to eat under the floating white and pink petals, I was reminded that outdoor social gatherings are vitally important to the well-being of city dwellers regardless of public health conditions. 