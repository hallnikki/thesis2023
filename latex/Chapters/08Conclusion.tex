The significance of public parks to public crises is not a new revelation. Time and again crises have uncovered new uses for parks, and the reasons to create new parks and adapt existing ones have only increased. The results from this research show the inadequacies of parks for a crisis that we have already experienced, but also demonstrate a need to be more proactive moving forward in preparing for inevitable future crises. Just as the surrounding urban environments have changed dramatically since the parks first opened, the park design can also change in a way that better serves today's needs, both in times of crises and times of normalcy. Before the next pandemic occurs, governments should be doing more to rectify these issues to be ready for the inevitable.

\begin{multicols}{2}
\section{Social gatherings}
The actions taken by public health officials in the first weeks of the COVID-19 pandemic were not unwarranted---a new was disease spreading throughout the world, straining hospitals and causing many fatalities. Lockdowns and stay-at-home orders curbed spread while medical professionals worked to understand the nature of the disease to prevent as many infections as possible. However, as the pandemic shifted from an emergency event to a years-long crisis, the urban environment could not stay shuttered, and living with a spreading disease became a new challenge. Because social interactions could not stay online for the duration of the pandemic, learning how to safely meet people with two-meters of social distance in a dense city was one of the challenges to overcome.

When life is normal and the weather is nice, people spend time in parks despite the crowds or perhaps even because of the crowds. When life is in peril, or at least sub-optimal as during a pandemic, people spend time in parks because in those conditions being around others can bring normality to an abnormal situation. The painted circles creatively implemented by Domino Park's director in May 2020 displayed how a social gathering can become a quantifiable unit of measure. Encouraging people to maintain social interactions while also preventing the spread of disease is a delicate balance, but total isolation is not the solution. Instead, safe spaces and guidelines can be issued by public health officials to encourage residents to meet social needs while living through a long-term public health crisis.

\section{Image classification results}
To quantify social gathering circles using the dimensions from the Domino Park case study, feasible locations for gathering circles are determined using image classification. Differences in climate mean that the conditions for an ideal gathering spot in one park may differ from that in another park. In the case of the parks studied here, London and New York are generally cooler, so sunny spots are preferred, whereas Tokyo's climate ranging from "comfortable" to "very warm" means that shaded gathering spots are more necessary. Based on the image classification, it would appear that Hyde Park has the greatest number of gathering circles based on the percentage of grass lawn. In actuality, though, Tokyo's climate conditions and the prevalence of forested area in Yoyogi Park adds up to more usable gathering space when comparing the percentage of gathering circles to overall park area. 

\section{Reach analysis results}
Because public transportation was discouraged during the COVID-19 pandemic, it is important to consider park accessibility on foot for the situations when other options cease to exist. When comparing the population catchment for straight-line distance (to the park border) versus network distance (to a social gathering circle) for each of the parks, the discrepancies vary due to entry points and surrounding geographical features of each case study. Hyde Park and Yoyogi Park see the largest decrease in accessibility due to adjacent non-park green spaces which act as an obstacle for the residential areas beyond. Prospect Park has the highest percentage of the population catchment from the straight-line distance to the network distance, with 57 percent of the total one-kilometer catchment of residents able to access a gathering circle within one-kilometer network distance.

Understanding the effectiveness of a park---the accessibility of its entrances relative to population and amenities, and how different land cover categories are composed within the park---is hard to discern without a method like reach analysis. Several iterations of reach analysis were performed in this research, beginning with population per park entry point and gathering circles per entry point. The results from this analysis show that, with an average of one gathering circle for every resident living within one-kilometer network distance, Hyde Park has the most plentiful ratio of population to gathering circles. The ratio for Yoyogi Park is on average one gathering circle for every three residents, an ideal value because each circle can hold one to four individuals while adhering to social distancing recommendations. Prospect Park only has one circle for every nine residents, insufficient for providing gathering circles to all residents at once, although unlikely, when compared against Yoyogi Park and Hyde Park, could be improved.

When evaluating the reach analysis results for food amenities around the parks, Prospect Park has the lowest reach values for food amenities from the gathering circles within the park, due to both the size of the park and the location of the amenities relative to the park borders. Gathering circles within Yoyogi Park have high reach values to the east and south, but in the area closest to the greatest residential population, the number of food amenities within reach decreases. Hyde Park, on the contrary, has fairly evenly distributed of amenities, with more than one-hundred food amenities within a one-kilometer walk of most of the gathering circles. 

In conclusion, each park can be improved but the strengths should also be acknowledged, so planners and designers can learn from what is already established and effective. Hyde Park has evenly distributed entry points and food amenities, but many of its gathering circles are grouped on the east side of the park. One remedy would be to evaluate the sporting facilities on the south edge of the park to evaluate the relevance of that programming and decide whether those areas could be made available for gathering circles during a pandemic. 

For Prospect Park, the programming should be studied to see where more areas for social gatherings can be created. For example, the large artificial water features might be better utilized if, at least partially, turned into lawns for gathering areas. Also, unlike Yoyogi Park, the forested areas of Prospect Park are not used for social gatherings due to the unmaintained conditions. If areas of the existing ground under tree canopies is made approachable for visitors, then social gathering circles could be quantified in those zones, increasing the total park count. Finally, changes in zoning might allow restaurants to operate closer to the park, creating a better relationship between the park and the surrounding neighborhood.

Finally, Yoyogi Park has many social gathering opportunities and many residents living around the park, but the entry points are few and far-between. The road networks around Yoyogi Park could be analyzed in more detail to study whether the main thoroughfare cutting through the south edge of the park is still serving the area in a positive way. Perhaps if this road could be redesigned, an increase in entry points would create a more comfortable experience walking from the neighborhood into the park.

\section{Future considerations}
We need to consider that parks should serve residents in normal times, but should also be planned and designed in a way that keeps them operational in crisis. This research evaluated one particular kind of pandemic with baseline assumptions about the way in which the disease is spread. To be truly prepared for the future, urban environments should be ready not only for another pandemic, but other emergency events as well, making it easier to quickly adapt our parks in order to curb the effects of the crisis at hand. And although social interaction is not an immediate necessity in an emergency, it remains a vital piece of our well-being during long-term crises and its importance should not be forgotten.

\end{multicols}
